%%%%%%%%%%%%%%%%%%%%%%%%%%%%%%%%%%%%%%%%%%%%%%%%%%%%%%%%%%%%%%%%%%%%%%%%%%%%%%%%%%%%%%%%%%%
% Original author of template for CV (Plasmati Graduate CV v1.0 24/03/2013):
% Alessandro Plasmati (alessandro.plasmati@gmail.com)
% Changes from the template onwards by (latest 30/01/2019):
% Pol del Aguila Pla (poldap@kth.se)
% Downloaded from:
% http://www.LaTeXTemplates.com
% License:
% CC BY-NC-SA 3.0 (http://creativecommons.org/licenses/by-nc-sa/3.0/)
% Important note (PdaP):
% This template needs to be compiled with XeLaTeX.
% The main document font is called Fontin and can be installed in sudo-enabled
% Linux systems by running fixFONTproblem.sh (will internally call sudo when necessary).
%%%%%%%%%%%%%%%%%%%%%%%%%%%%%%%%%%%%%%%%%%%%%%%%%%%%%%%%%%%%%%%%%%%%%%%%%%%%%%%%%%%%%%%%%%

% ----------------------------------------------------------------------------------------
%	PACKAGES AND OTHER DOCUMENT CONFIGURATIONS
% ----------------------------------------------------------------------------------------

% Font and paper size
\documentclass[a4paper,10pt]{article}

% Font loading
% \usepackage{fontspec}
% \defaultfontfeatures{Mapping=tex-text}
% Set main font for document
% \setmainfont[SmallCapsFont = Fontin SmallCaps]{Fontin}

% Formatting
% \usepackage{xunicode,xltxtra,url,parskip}

% Coloring
\usepackage[usenames,dvipsnames]{xcolor}

% Margin specification
\usepackage{fullpage}

% Links and other clickable references
\usepackage{hyperref}
% Link colors
\definecolor{linkcolour}{rgb}{0,0.2,0.6}
\hypersetup{colorlinks,breaklinks,urlcolor=linkcolour,linkcolor=linkcolour}

% Costumize section command
\usepackage{titlesec} % Used to customize the \section command
% Text formatting
\titleformat{\section}{\Large\scshape\raggedright}{}{0em}{}[\titlerule]
% Spacing
\titlespacing{\section}{0pt}{3pt}{3pt}

% Insert images
\usepackage{graphicx}

% Footnotes in tables
\usepackage{footnote}

% Tables that can span several pages
\usepackage{longtable}

% Several biblographies
\usepackage{bibunits}

% Trademark symbol
\usepackage{textcomp}

% Put references sections in the TOC (PDF and HTML links)
\usepackage[nottoc,numbib]{tocbibind}

\def\myname{Liu Dong}


\begin{document}

% Remove page numbers
\pagestyle{empty}

% ----------------------------------------------------------------------------------------
%	NAME AND CONTACT INFORMATION
% ----------------------------------------------------------------------------------------

\begin{center}
  \begin{tabular}{lcr}
    \par{\centering{\Huge Dong Liu}\bigskip\par} & & %\includegraphics[width=0.3\textwidth]{../main/pol.jpg} \\
  \end{tabular}
\end{center}

\section{Personal data}

\begin{tabular}{rl}
  \textsc{Address:} & Malvinas Vag 10, Stockholm, Sweden \\
  \textsc{Phone:} & +460767103249 \\
  \textsc{Email:} & \href{mailto:doli@kth.se}{\nolinkurl{doli@kth.se}} \\
  % \textsc{Google Scholar:} & \href{https://scholar.google.com/citations?user=eK9LoQMAAAAJ&hl=en}{\aiGoogleScholar}\\
  \textsc{Website:} & \href{https://firsthandscientist.github.io}{https://firsthandscientist.github.io}\\
\end{tabular}

% ----------------------------------------------------------------------------------------
%	Work
% ----------------------------------------------------------------------------------------

\section{Work Experience}
\begin{tabular}{r|p{13cm}}
  \textsc{Present}  & Researcher on Bayesian methods, Department of Intelligent Systems \\
  \textsc{2020 Nov}  & KTH Royal Institute of Technology, Stockholm, Sweden.\\
  \multicolumn{2}{c}{} \\

  \textsc{2020 Jan}  & Consultant of start-up, \href{http://www.siajor.com/}{SIAJOR} Co. Ltd \\
  \textsc{2018 Jan}  & A solution provider of artificial intelligent systems. \\
  \multicolumn{2}{c}{} \\

  \textsc{2016 Dec}  & Engineer, standardization of radio access technology  \\
  \textsc{2016 Apr}  & Shanghai Research Institute, Huawei Technologies Co., Ltd \\
\end{tabular}


% ----------------------------------------------------------------------------------------
% EDUCATION
% ----------------------------------------------------------------------------------------

\section{Education}

\begin{tabular}{r|p{13cm}}

  \textsc{2020 Nov} 	& Ph.D. (temporary employee)\\
  \textsc{2016 Dec} 	& {Research in statistical models, Bayesian inference, machine learning} \\
                      & {Dissertation: \href{http://urn.kb.se/resolve?urn=urn:nbn:se:kth:diva-284140}{\textbf{Perspectives on Probabilistic Graphical Models}}} \\
                      & { Information Science and Engineering, Department of Intelligent Systems
                        } \\
                      & {School of Electrical Engineering and Computer Science} \\
                      & {KTH Royal Institute of Technology, Stockholm, Sweden.} \\
  \multicolumn{2}{c}{} \\

  % ------------------------------------------------

  \textsc{2016 Mar} 	& M.Sc. \\
  \textsc{2013 Sep}  & \textbf{Top $\mathbf{5\%}$, Excellent Graduate of Shanghai City in 2016.}\\
                      & {Department of Information and Communication Engineering} \\
                      & {School of Electrical and Information Engineering} \\
                      & {Tongji Univsersity, Shanghai, China.}\\
  \multicolumn{2}{c}{} \\

  % ------------------------------------------------

  \textsc{2013 Jul} 	& B.E. \\
  \textsc{2009 Sep} & \textbf{Top $\mathbf{1\%}$, Excellent Graduate of Liaoning Province in 2013}\\
                      & {Department of Information and Communication Engineering}\\
                      & {Shenyang University of Technology, Shenyang, China} \\
  \multicolumn{2}{c}{}\\

\end{tabular}


\section{Teaching}
\begin{tabular}{r|p{13cm}}
  \textsc{2020 Fall}  & Teaching Assistant \\
                      & Industry course, {EP232U Deep Neural Networks}, KTH Royal Institute of Technology, Sweden.\\
  \multicolumn{2}{c}{} \\

  \textsc{2020 Sep}  & Teaching Assistant \\
  \textsc{2017 Sep}  & Graduate course \href{https://www.kth.se/student/kurser/kurs/EQ2341?l=en}{EQ2341 Pattern Recognition and Machine Learning} (regularly offer once per year), KTH Royal Institute of Technology, Sweden.\\
  \multicolumn{2}{c}{} \\

  % ------------------------------------------------
  \textsc{2020 Sep}  & Supervision of Master Students at KTH \\
  \textsc{2018 Jan}  & A. Scotti (topic: graph neural network \& approximate message passing), graduated \\
                      & A. Ghosh (topic: neural network \& hidden Markov model), graduated \\
                      & J. Torres (topic: approximate inference \& message passing), graduated
\end{tabular}

\section{Summer School}
\begin{tabular}{r|p{13cm}}

  \textsc{2012 Sep}  & Chinese Academy of Sciences, China. \\
                     & Scholarship provided by the Chinese Academy of Sciences. \\
  \multicolumn{2}{c}{}\\
  \textsc{2012 Aug}  & Kochi University of Technology, Kochi, Japan. \\
                     & Scholarship provided Kochi University of Technology.
\end{tabular}

\section{Research Interests}
Machine learning, Bayesian inference, graphical models, stochastic geometry and their applications. Click \href{https://firsthandscientist.github.io/#/research}{R\&D} to see more about my recent project information.
% ----------------------------------------------------------------------------------------
%	PUBLICATIONS
% ----------------------------------------------------------------------------------------

\begin{bibunit}[IEEEtran_Pol]
  \renewcommand\refname{Recent Publications (See more at \href{https://scholar.google.com/citations?hl=en&user=eK9LoQMAAAAJ&view_op=list_works&sortby=pubdate}{my Scholar})}
  \nocite{
    liu2020region,
    liu2020alpha,
    zuxing2020cdc,
    anubhad2020,
    andrea2020,
    liu2020powering,
    honore2020hidden,
    liu2020neural,
    chatterjee2019ssfn,
    liu2019discontinuous,
    liu2019entropy,
    liu2019alpha,
    liu2019dominant,
    liu2018will}
  % liu2015bounds,
  % wang2015rsh,
  % zhang2015effective,
  % liu2015node,
  % wang2015structural,
  % liu2014secondary,
  % liuenergy,
  % ren2015applying,
  % ren2015spectram,
  % ren2015spectrum

  \footnotesize{\putbib[../bibfile]}
\end{bibunit}
% ----------------------------------------------------------------------------------------
% LANGUAGES
% ----------------------------------------------------------------------------------------

\vspace{-10pt}

\section{Languages}

\begin{tabular}{rp{10cm}}

  \textsc{Mother tongue :} & Chinese \\

  \textsc{Professional :} & English \\

\end{tabular}

% ----------------------------------------------------------------------------------------
% COMPUTER SKILLS
% ----------------------------------------------------------------------------------------

\section{Practical skills}

\begin{tabular}{rp{12cm}}
	Technical skills:  & I am a superuser. \\
                     & Experience in the administration of Linux computational servers (2017-2020).
                       \vspace{5pt}\\
	Programming:       & \textsc{Python, Pytorch, Bash, Matlab, \LaTeX}  \vspace{5pt}\\
\end{tabular}

% ----------------------------------------------------------------------------------------
%	SCHOLARSHIPS AND ADDITIONAL INFO
% ----------------------------------------------------------------------------------------
\section{Grants \& Scholarships}
\begin{savenotes}
  \begin{longtable}{r|p{13cm}}
    % General Travel grant: 21500 SEK
    % Ericsson research grant: 20 000 SEK
    % Wallenberg jubilee 10100 SEK
    % SPS 500 USD = 4846 SEK
    % Karl Engvers foundation = 27 000 SEK
    % Current sum: 83446 SEK
    \textsc{2019 Nov} & Grants, amount $\approx 83k\,\mathrm{SEK}$.\\
    \textsc{2017 Jan} & {} \\
                      & Grant from Karl Engvers Foundation, Sweden, 2020. \\

                      &  Grant from Knut and Alice Wallenberg Foundation
                        "Jubilee appropriation", Sweden, 2019. \\
                      & Grant from Ericsson Research Foundation, Sweden, 2019. \\
                      & Grant from General Travel Foundation, KTH, Sweden, 2019. \\
                      & Gran from IEEE Signal Processing Society Travel Grant, 2019. \\
    \multicolumn{2}{c}{} \\
    % National graduate scholarship: 8k * 2.5 = 20k RMB
    % National Scholarship for Postgraduates in 2015: 20k RMB
    % period amount: 40k RMB = 55k SEK
    \textsc{2016 Dec} & Postgraduate studies scholarships, amount $\approx 55k\,\mathrm {SEK}$. \\
    \textsc{2013 Aug} & \\
                      & National Scholarship for Postgraduate Studies, China, 2013-2016. \\
                      & National Scholarship for Postgraduate, China, 2015.\\
    \multicolumn{2}{c}{} \\

    % National schorlarship: 8k *2 RMB
    % Liaoning provincial: 8k RMB
    % mayor: 6k RMB
    %
    % Special Scholarship from Shenyang University of Technology in 2012 and 2011, respectively: 2K *2 RMB
    % First-class Scholarship from Shenyang University of Technology in 2010: 1.5 K RMB
    % period amount: 35.5K RMB = 48.9K RMB
    \textsc{2013 Jun}	& Undergraduate studies scholarships, amount $\approx 49k\,\mathrm {SEK}$.\\
    \textsc{2009 Sep}	& \\
                      & National Scholarship for Undergraduates in 2012 and 2011, respectively.\\
                      & The First Class Scholarship of Chinese Instrument and Control Society in 2012.\\
                      & The Mayor Scholarship of Shenyang City in 2011.\\
                      & The Scholarship of Liaoning Provincial Government in 2010.\\
                      & The Special Scholarship in 2012, 2011 and the First-class scholarship in 2010, from Shenyang University of Technology.\\
    \multicolumn{2}{c}{} \\

  \end{longtable}
\end{savenotes}

% ----------------------------------------------------------------------------------------
%	Awards and Hornors
% ----------------------------------------------------------------------------------------
\section{Contest Awards}

\begin{longtable}[H]{r|p{13.5cm}}
  \hspace{20pt} \textsc{2019} & The Bronze Award in the $5$th China Internet+ University Graduates Innovation \& Entrepreneurship Awards. \\
  \multicolumn{2}{c}{} \\
  \textsc{2014} & The First Prize in the National Postgraduate Mathematic Contest in Modeling in China.\\
                              & Top \textbf{$\mathbf{2.45\%}$ in 4900} teams in China, fast fading channel modeling and optimization. Algorithm optimization and programming for channel modeling simulation.\\
  \multicolumn{2}{c}{} \\

  \textsc{2012} & The President Award in Fukuda Gold Robot Cup Contest of Shenyang City. \\
                              & My team spent 2 months designing a searching robot capable of seeking and picking metal disks in a given area. My work: circuit welding and a part of programming for the SoC. \\
  \multicolumn{2}{c}{} \\

  \textsc{2011} & The Grand Prize in Liaoning Contest District of National Undergraduate Electronic Design Contest. \\
                              & A digital signal transmission analyzer was designed by my team. I programmed the SoC in this contest.\\
  \multicolumn{2}{c}{} \\

  \textsc{2011} & The First Prize in Liaoning Contest District of Chinese Undergraduate Mathematical Contest in 2011 and
                  2010 respectively. \\
\end{longtable}

\section{Honors}
\begin{longtable}[H]{r|p{13.5cm}}
  \textsc{2016 Jun} & The Excellent Graduate of Shanghai City in 2016. \\
  \textsc{to}& The Excellent Graduate of Liaoning Province in 2013. \\
  \textsc{2011 Sep} & The Excellent Graduate of Shenyang University of Technology in 2013. \\
                    & The Award Nomination in People of Year 2012 of Liaoning Provincial Undergraduates. \\
                    & The Pivot of Merit Students of Liaoning Province in 2012. \\
                    & The Outstanding Inspirational Talent of College Students of Liaoning Province in 2011. \\
                    & The Youth Medal of Shenyang University of Technology in 2012 (4 award winners are selected among over 16,000 undergraduates every two years). \\
                    & The {Second Prize for Outstanding Contribution} to the university in 2011, the {Excellent Student Leader} in 2011, the {Pivot of Merit Students} in 2010, the {Top Ten Students} of school in 2012, 2011 and 2010, respectively, Shenyang University of Technology.
\end{longtable}

% ----------------------------------------------------------------------------------------
%	Contact with the scientific community
% ----------------------------------------------------------------------------------------

% \section{Participation in the scientific community}

% \begin{longtable}[H]{r|p{13.5cm}}
%   \emph{Current}	& Reviewer for \textbf{Elsevier Signal Processing} \\
%   \textsc{2019 Mar} &  \\
%   \multicolumn{2}{c}{} \\

%   \emph{Current}	& Reviewer for the \textbf{IEEE Transactions on Signal Processing} \\
%   \textsc{2015 Aug} &  \\
%   \multicolumn{2}{c}{} \\


%   \textsc{2019 Apr} &  \textsc{12-13 Apr}, Invited participant in the 2019 IEEE Signal Processing Society (SPS) Long Range Planning Meeting\\
%   & \\
%   & Lecture presentation (\textsc{9 Apr}) within the \emph{2019 IEEE $16^{\mathrm{th}}$ International
%   Symposium on Biomedical Imaging} (\textbf{ISBI 2019}), titled
%   \textbf{SpotNet --- Learned iterations for cell detection in image-based immunoassays}, access at \href{https://embs.papercept.net/conferences/scripts/myprogram.pl?ConfID=79&Add=208}{\texttt{embs.papercept.net}}.\\
%   & \footnotesize{Hilton Molino Stucky, Venice, Italy.} \\
%   \multicolumn{2}{c}{} \\



%   \textsc{2019 Jan} & \textsc{7 Jan - 7 Feb}, Research visit at \textbf{Professor Jean-Luc Starck}'s group (\textbf{CosmoStat}). \\
%   & \footnotesize{Department of Astrophysics, \textbf{CEA Paris-Saclay}, Paris, France} \\
%   & \\
%   & \textsc{14 and 28 Jan}, presentations at \textbf{Cosmostat} and \textbf{Parietal}, \textbf{NeuroSpin}, \textbf{INRIA}, respectively, titled \emph{Cell detection by functional inverse diffusion and non-negative group sparsity - Biology, physics, math and engineering}, access at \href{http://www.cosmostat.org/events/cosmoclub/cosmosclub-pol}{\texttt{www.cosmostat.org}} and \href{https://team.inria.fr/parietal/slides-of-pol-del-aguila-plas-talk-available-now-online/}{\texttt{team.inria.fr/parietal}}. \\
%   \multicolumn{2}{c}{} \\

%   \textsc{2018 Jul} & Attendance to the \emph{Thirty-fifth International Conference on Machine
%   Learning} (\textbf{ICML 2018}). \\
%   & \footnotesize{Stockholmsm\"{a}ssan, Stockholm, Sweden.} \\
%   \multicolumn{2}{c}{} \\

%   \textsc{2018 Jun} & Poster presentation within the \emph{SIAM Conference on Imaging Science}
%   (\textbf{SIAM-IS 2018}), titled \textbf{Source localization by spatially
%   variant blind deconvolution}, access at \href{https://www.siam-is18.dm.unibo.it/uploads/store/a7a8b242b168225d0be8998fa373f58b.pdf}{\texttt{www.siam-is18.dm.unibo.it}}. \\
%   & \footnotesize{\textbf{University of Bologna}, Bologna, Italy.}\\
%   \multicolumn{2}{c}{} \\

%   \textsc{2018 Apr} & Poster presentation within the \emph{IEEE International Conference
%   on Acoustics, Speech and Signal Processing} (\textbf{ICASSP 2018}), titled
%   \textbf{Convolutional group-sparse coding and source localization}, access at \href{https://sigport.org/documents/convolutional-group-sparse-coding-and-source-localization}{\texttt{sigport.org}}. \\
%   & \footnotesize{Calgary Talus Convention Centre, Calgary, Alberta, Canada.} \\
%   & \\
%   & \textsc{9 - 13 Apr}, Research visit at \textbf{Professor Stephen P. Boyd's group}. \\
%   & \textsc{10 Apr}, Presentation to the group titled
%   \emph{Cell detection by functional inverse diffusion and non-negative group sparsity}.\\
%   & \footnotesize{Information Systems Laboratory, Department of \textbf{Electrical Engineering,
%   Stanford University}, Stanford, California, United States of America} \\
%   & \\
%   & Poster presentation within the \emph{2018 IEEE $15^{\mathrm{th}}$ International
%   Symposium on Biomedical Imaging} (\textbf{ISBI 2018}), titled
%   \textbf{Cell detection on image-based immunoassays}.\\
%   & \footnotesize{Omni Shoreham Hotel, Washington, D.C., United States of America.} \\
%   \multicolumn{2}{c}{} \\

%   \textsc{2017 Nov} & Lecture presentation within the workshop \emph{Generative models,
%   parameter learning, and sparsity} (VMVW02), titled
%   \textbf{Cell detection by functional inverse diffusion and group sparsity},
%   access at \href{https://downloads.sms.cam.ac.uk/2600830/2600858.mp4}{\texttt{downloads.sms.cam.ac.uk}}. \\
%   & \footnotesize{\textbf{Isaac Newton Institute for Mathematical Sciences},
%   \textbf{University of Cambridge}, Cambridge, United Kingdom.
%   Within the programme \emph{Variational methods and effective
%   algorithms for imaging and vision}.}\\

% \end{longtable}

% ----------------------------------------------------------------------------------------
% TEACHING EXPERIENCE
% ----------------------------------------------------------------------------------------
% \section{Teaching experience}

% \begin{bibunit}[IEEEtran_Pol]
%   \begin{tabular}{r|p{13cm}}

%     \emph{Current}	 & Supervision of Master's and Bachelor's thesis (see below) \\
%     \textsc{2017 Feb}  & \footnotesize{Main supervisor of two master's thesis, \cite{Jones2018} and one
%     ongoing.
%     Designer of two bachelor's thesis projects, attracting $5$
%     different groups of two students. Main supervision for $2$ bachelor's
%     thesis \cite{F2A2018,F2B2018}, and co-supervisor for six,
%     \cite{F3A2018,F3B2018,F3C2018} and three ongoing. } \\
%     \multicolumn{2}{c}{} \\

%     \textsc{2018 Dec}     & Teaching assistance in the course \textbf{EQ2300: Digital Signal Processing} \\
%     \textsc{2014 Sep} & \footnotesize{Taught every year \textsc{Nov -- Dec} by Prof. Joakim Jald\'{e}n and
%     one to three assistants. Approximate numbers: $120\,\mathrm{h}$ of guidance of
%     exercise sessions and lectures, $140\,\mathrm{h}$ of
%     class preparation, $45\,\mathrm{h}$ of course and material development,
%     $100\,\mathrm{h}$ of grading of projects and exams, and $20\,\mathrm{h}$ of private tutoring.} \\

%   \end{tabular}


%   \renewcommand\refname{\normalsize{Supervised theses}}
%   \footnotesize{
%   \putbib[../pubs/bibfile_students]}
% \end{bibunit}


% ----------------------------------------------------------------------------------------
% REFEREE
% ----------------------------------------------------------------------------------------
\section{References}
\begin{longtable}[H]{r|p{13.5cm}}
  Assoc. Prof. & School of Electrical Engineering and Computer Science\\
  Ragnar Thobaben &Royal Institute of Technology (KTH), Sweden \\
               & Office: Room C:738, Malvinas vag 10, Stockholm, SE-100 44, Sweden \\
               & Email: ragnart@kth.se \\
               & Phone: +4687908452 \\

  \multicolumn{2}{c}{} \\
  Assoc. Prof. & School of Electrical Engineering and Computer Science\\
  Saikat Chatterjee &Royal Institute of Technology (KTH), Sweden \\
               & Office: Malvinas vag 10, Stockholm, SE-100 44, Sweden \\
               & Email: sach@kth.se \\
               & Phone: +4687908478 \\

  \multicolumn{2}{c}{} \\

  Prof.  & School of Electrical Engineering and Computer Science \\
  Mikeal Skoglund &Royal Institute of Technology (KTH), Sweden \\
               & Office: Room A:415, Malvinas vag 10, Stockholm, SE-100 44, Sweden \\
               & Email: skoglund@kth.se \\
               & Phone: +4687908430 \\

  \multicolumn{2}{c}{} \\
  Prof. & School of Electronics and Information Engineering \\
  Erwu Liu &Tongji University, China \\
               & Office: 4800 Cao'an Highway, Jiading, 201804 Shanghai, China\\
               & Email: erwu.liu@ieee.org or erwuliu@tongji.edu.cn \\
               & Phone: +8615800903275
\end{longtable}



\end{document}
%%% Local Variables:
%%% mode: latex
%%% TeX-master: t
%%% End:
